\chapter{システムの提案}
\label{chap:proposal}

本章では,音声利用についての背景を踏まえ、関連研究を概観し、新しい録音システム「Gyaon」を提案する。

\newpage

\section{関連研究およびアプリケーション}

\subsection{雰囲気の記録}

Poupyrevらは、日常的な体験や雰囲気を記録するための録音デバイスを提案している\cite{Poupyrev}。
録音は何かを記録、説明したり、思い出すのに重要な手段だが、日常的に行っている人は少ないと指摘しており、
日用品にセンサーを埋め込み、状況の変化に応じて録音/再生するデバイスを設計した。
日用品に溶け込み、単純明快な録音/再生インターフェースが実現されているが、
本を閉じる/開くといった二値的な状況変化でしか操作できないため、
デバイス単体で複数の音声を管理することが困難になっている。

\subsection{長時間音声データの管理}

長時間にわたる音声データでは、聞き返したい箇所を効率良く見つけることが困難となる。
この課題に対して、音声データのブラウジングやタグ付けの手法について研究が行われている。

\paragraph*{ブラウジング}

人間生活に関わる情報を長期間にわたり記録するライフログ関連研究では、
そのログの一部として音声データがよく利用されている\cite{Bell}。
音声データの特徴量を計算し,閾値処理するインターフェースを提供し、再生区間を決定する手法\cite{Kawamura}や、
講義ノートといった手書きメモに対して音声データを関連付ける手法\cite{Stifelman}などが検討されている。

\paragraph*{タグ付け}

研究ノートの補助的な記録手段として音声データを利用するシステムが提案されている\cite{Kawanishi}。
事前に用意された研究ノートから形態素解析などを利用してキーワードを抽出し、音声データに付与する手法が検討されている。

また、音声データの特定のタイミングにタグを付与できる録音システムが開発されている\cite{Fujisaka}。
学生がノートテイキングの補助として使うことを想定しており、重要な説明等を逃さないよう素早くタグ付けが行えるインターフェースが実装されている。

\subsection{音声ログの検索手法}

音声認識技術を活用した、テキストによる音声検索手法が提案されている\cite{Vemuri}。
音声データから単語を抽出し、キーワード検索が可能となっている。

\subsection{写真との組み合わせ}

音声と写真を関連付けて利用することで、〜する研究がある\cite{Nakakura}\cite{Masui}

\section{設計方針}

TBD

\section{利用例}

TBD
