\chapter{システムの提案}
\label{chap:proposal}

本章では,音声利用についての背景を踏まえ,関連研究を概観し,新しい録音システム「Gyaon」を提案する.

\newpage

\section{設計指針}

「Gyaon」では
\begin{itemize}
\item 単純な録音操作
\item 音声管理の簡単さ
\item 他システムからの音声の利用のしやすさ
\end{itemize}
を優先した設計を行うことで録音の不便を解消し、音声の有効活用を促すシステムを目指している。

\section{利用例}
Gyaonは様々な利用シーンに対応するため、PC用Webアプリケーション、
Androidスマートフォン用アプリケーション、
およびその他の録音に特化したハードウェア
にてシステム構築する。
それぞれのプラットフォームにおける利用例を述べる。

\subsection{PC版}

%基本操作
%録音操作
%
%再生操作
%
%コメントの追加
%
%ダウンロードリンクの取得
%
%プリレコーディング機能
%
%地図機能
%
%IDの共有


% 録音即再生・アップロード
% ボタンを押している間録音する
% 離したら録音停止して、すぐサーバにアップロードされる
% マウスオーバーで再生
%  録音した音声はリストに表示され、マウスカーソルを重ねるだけで再生できる
% 音声にコメントを付けられる
% 音声の直リンクを取得できる
%  他人にすぐ共有できる
%  ダウンロードすればローカル環境で利用できる
% さかのぼり録音できる
%  今いいこと言った!を逃さない
%  急に面白い音が聞こえてきても逃さない
% 音声を地図上にマッピングできる
%  どこで撮ったのか[わかる.icon]
% ユーザIDを共有してコミュニケーションする
%  誰かが録音するとすぐに再生される
% Gyaonキー
\subsection{Android版}
% PCの無い場所や外出先でもWeb版と同様に録音可能
% 画面に録音ボタンを常駐させておけるので録音が簡単
\subsection{その他}
% 録音ペダル(録音専用コンピュータ)
%  [satake 2016/10/24発表資料]
%  ばら撒いて使う
%  ハンズフリーで録音したいときに有用
%   楽器演奏
%   運転中
%   料理中?
