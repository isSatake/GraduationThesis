\chapter{結論}
\label{chap:conclusion}

本章では,本研究の成果と総括について述べる.

\newpage

\section{研究の成果}

本研究では,既存のリアルタイム情報共有システムの問題点を分析し,
情報ダッシュボード表示を用いた情報共有システム『わかるらんど』の開発を行った.

まず第2章において,現在のリアルタイム情報共有インタフェースの主流であるタイムライン表示の分析や
その他にも用いられている情報表現の工夫について紹介した.

第3章では,本研究に関連する研究を紹介し,
それぞれのアプローチの特徴と問題点を分析した.

第4章では,第2章でのリアルタイム情報共有における情報表現の工夫の分析をもとに,
本研究で開発したソフトウェア『わかるらんど』のインタフェースとねらいについて述べた.

第5章では,IoT時代の共有情報視覚化システム『わかるらんど』の詳細な実装について述べた.
ここでは,システムの構成,使用したプログラミング言語,プロセス間通信アーキテクチャなどを述べた.

第6章では,日本ソフトウェア科学会主催のWISS2016コンファレンスにおいての『わかるらんど』の評価実験と
9ヶ月にわたる筆者の研究室での運用について述べた.

第7章では,WISS2016での評価実験と研究室での長期運用をもとに,
『わかるらんど』システムの有効性について議論した.

\section{総括}

本研究では,情報ダッシュボード形式の視覚化システム『わかるらんど』の開発を行った.

『わかるらんど』は単純なアーキテクチャながら,ユーザの気分を表明したり,
チャット文字列を投稿したり,センサ情報やWeb上の情報を表示したり,
ネット上のあらゆる情報を投稿/共有して一覧表示することできる.

会議などで特定の人が沢山発言する状況を,本システムを使うことで,
誰もが気軽に意見を表明できる環境を構築できる可能性を示した.
また,ネット上のあらゆる情報を投稿/共有する
IoT時代の新しい情報視覚化手法として本システムが有益であることを示した.