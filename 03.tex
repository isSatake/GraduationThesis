\chapter{実装}
\label{chap:implementation}

本章では,第2章で述べたシステムの設計を受け,Gyaonの実装について述べる.

\newpage

\section{システム構成}
Gyaonは,ユーザが実際に録音/再生するためのクライアントアプリケーションと,
アップロードされた音声を保存/管理するサーバから構成される.構成図を図\ref{}に示す.


%サーバ
%    ストレージ
%    DB
%クライアント
%    PC
%    Android

%簡単な図を描きたい

\section{サーバ}
サーバはNode.js\footnote{\textsf{https://nodejs.org}}
とそのWebアプリケーションフレームワークであるExpress\footnote{\textsf{http://expressjs.com/ja/}}
によって実装されている.音声データの保存をAmazon S3
\footnote{\textsf{Amazon Web Servicesによって提供されるオンラインストレージサービス.https://aws.amazon.com/jp/s3/}},
DB管理をmLab\footnote{\textsf{MongoDBベースのクラウドDBサービス.https://mlab.com/}}にて行っている.

\subsection{DBスキーマ}

DBでは以下のようなスキーマを定義し,音声データを管理している.

\vspace{4mm}
\begin{lstlisting}
var soundSchema = mongoose.Schema({
    lastmodified: Date,     /* 更新時刻 */
    user: String,           /* ユーザID */
    name: String,           /* ファイル名 */
    key: String,            /* S3key */
    size: Number,           /* ファイルサイズ */
    time: Number,           /* 再生時間 */
    comment: String,        /* コメント */
    location_x: Number,     /* 位置情報 */
    location_y: Number
})
\end{lstlisting}

\section{PC版クライアント}
PC版クライアントはHTML/CSS/JavaScriptで実装されており,ブラウザ上のWebアプリケーションとして動作する.
%各機能の実装について述べる.
%特筆すべき実装
%詳細な実装

\subsection{WebAudioAPI}
%WebAudioAPIを解説する
録音/再生といった主要な音声機能は,
HTML5 audio\footnote{\textsf{https://www.w3.org/TR/html5/embedded-content-0.html\#the-audio-element}}
並びにWebAudioAPI\footnote{\textsf{https://www.w3.org/TR/webaudio/}}によって実装されている.

HTML5 audioはHTMLドキュメント内に音声を埋め込むことができ,以下のようなAudio要素を宣言することで利用できる.

\vspace{4mm}
\begin{lstlisting}
<audio src="/test/audio.ogg"></audio>
\end{lstlisting}

WebAudioAPIは高度な音声処理や音声合成を行えるJavaScriptAPIであり,
Audio要素を操作したり,エフェクトを加えることができる.

\subsection{音声リストの同期}
複数のユーザが同じユーザIDを使用していても,音声リストはリアルタイムに同期される.
これはNode.jsのWebSocketライブラリ「Socket.IO\footnote{\textsf{http://socket.io}}」によって実装されており,
サーバに音声がアップロードされると,音声リストの更新情報がブラウザに通知される.

\subsection{プリレコーディング機能}
%shift/pushしている部分のコードを貼る

%録音ボタンを押す前からの音声を撮れる
%音声を入れてる配列の操作で
%〜は,以下のように実装されている

\vspace{4mm}
\begin{lstlisting}
preRecScriptProcessor.onaudioprocess = (event) => {
    const channel = event.inputBuffer.getChannelData(0)
    if(preRecAudioBufferArray.length * BUFFER_SIZE > SAMPLE_RATE * PREREC_SEC) {
        preRecAudioBufferArray.shift()
    }
    preRecAudioBufferArray.push(new Float32Array(channel))
}
\end{lstlisting}

\subsection{Gyaonキー}
%Karabinerを使っている
%録音するシェルスクリプトを叩かせている
%スクリプト本体やkarabiner設定ファイルを貼る

\section{Android版クライアント}
Android版クライアントはJavaで実装されており,通常のAndroidアプリケーションとして動作する.
%常駐型の録音ボタンは,現状serviceを使っているが
%serviceで常駐
%常駐ボタンが邪魔だったりするので,他の方法を考えたい
%スマホを起動しなくてもいい方法
%    専用ボタンとか
%    ヘッドセットとか
