\chapter{実装}
\label{chap:implementation}

本章では,第2章で述べたシステムの設計を受け,Gyaonの実装について述べる.

\newpage

\section{システム構成}
Gyaonは,ユーザが実際に録音/再生するためのクライアントアプリケーションと,
アップロードされた音声を保存/管理するサーバから構成される.構成図を図\ref{}に示す.


%サーバ
%    ストレージ
%    DB
%クライアント
%    PC
%    Android

%簡単な図を描きたい

\section{サーバ}
サーバはNode.jsとExpressによって実装されている.
音声データの保存をAmazon S3
\footnote{\textsf{Amazon Web Servicesによって提供されるオンラインストレージサービス.https://aws.amazon.com/jp/s3/}},
DB管理をmLab\footnote{\textsf{MongoDBベースのクラウドDBサービス.https://mlab.com/}}にて行っている.

\section{PC版}
PC版GyaonはHTML/CSS/JavaScriptで実装されており,ブラウザ上のWebアプリケーションとして動作する.

\section{Android版}
Android版GyaonはJavaで実装されており,通常のAndroidアプリケーションとして動作する.
%常駐型の録音ボタンは,現状serviceを使っているが
%serviceで常駐
%常駐ボタンが邪魔だったりするので,他の方法を考えたい
%スマホを起動しなくてもいい方法
%    専用ボタンとか
%    ヘッドセットとか
