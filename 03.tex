\chapter{実装}
\label{chap:implementation}

本章では,第2章で述べたシステムの設計を受け,Gyaonの実装について述べる.

\newpage

\section{システム構成}
Gyaonは,〜からなる.構成図を図\ref{}に示す.

%サーバ
%    ストレージ
%    DB
%クライアント
%    PC
%    Android

%簡単な図を描きたい

\section{サーバ}
%音声データの保存にS3
%音声データの管理にmongoDB
%音声データのアップロード受付/DB操作/ストレージへの保存をサーバが担う

\section{PC版}
%PC版のGyaonクライアントはHTML/CSS/JavaScriptで実装されており,ブラウザ上のWebアプリケーションとして動作する.

\section{Android版}
%Android版のGyaonクライアントはJavaで実装されており,通常のAndroidアプリケーションとして動作する.
%常駐型の録音ボタンは,現状serviceを使っているが
%serviceで常駐
%常駐ボタンが邪魔だったりするので,他の方法を考えたい
%スマホを起動しなくてもいい方法
%    専用ボタンとか
%    ヘッドセットとか
