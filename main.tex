\newif\ifjapanese
\japanesetrue  % 論文全体を日本語で書く(英語で書くならコメントアウト)
\ifjapanese
  %\documentclass[a4j,twoside,openright,11pt]{jreport} % 両面印刷の場合。余白を綴じ側に作って右起こし。
  \documentclass[a4j,11pt]{jreport}                  % 片面印刷の場合。
  \renewcommand{\bibname}{参考文献}
  \newcommand{\acknowledgmentname}{謝辞}
\else
  \documentclass[a4paper,11pt]{report}
  \newcommand{\acknowledgmentname}{Acknowledgment}
\fi
\usepackage[dvipdfmx]{graphicx}
\usepackage{thesis}
\usepackage{ascmac}
\usepackage{graphicx}
\usepackage{multirow}
\usepackage{url}
\usepackage{latexsym}
\usepackage{here}
\usepackage{listings,jlisting}

\lstset{%
  language={C},
  basicstyle={\small\ttfamily\footnotesize},%
  breaklines=true,%
  identifierstyle={\small},%
  commentstyle={\small\itshape},%
  keywordstyle={\small\bfseries},%
  ndkeywordstyle={\small},%
  stringstyle={\small\ttfamily},
  frame={tb},
  breaklines=true,
  columns=[l]{fullflexible},%
  numbers=left,%
  xrightmargin=0zw,%
  xleftmargin=3zw,%
  numberstyle={\scriptsize},%
  stepnumber=1,
  numbersep=1zw,%
  lineskip=-0.5ex%
}
\bibliographystyle{jwiss}

%\bindermode  % バインダー用余白設定

% 日本語情報(必要なら)
\jclass  {卒業論文}                             % 論文種別
\jtitle    {Gyaon}    % タイトル。改行する場合は\\を入れる
\juniv    {慶應義塾大学}                  % 大学名
\jfaculty  {環境情報学部}               % 学部、学科
\jauthor  {佐竹 紘明}                       % 著者
\jhyear  {28}                                   % 平成○年度
\jsyear  {2016}                                 % 西暦○年度
\jkeyword  {音声, 録音}     % 論文のキーワード
\jproject{増井俊之研究会} %プロジェクト名
\jdate{2017年1月}

\begin{document}

\ifjapanese
  \jmaketitle    % 表紙(日本語)
\else
  \emaketitle    % 表紙(英語)
\fi

% 日本語のアブストラクト
\begin{jabstract}
小型の録音装置は古くから販売されており,またスマートフォンでは様々な録音アプリケーションを利用できるが,
音声メモを記録したり様々な音を録音したりして活用している人は少ない.
音声が広く活用されていないのは,録音の手間が煩雑であり,音の再生/検索が簡単でないためだと思われる.
本研究では,既存の録音システムの不便を解消し,音声活用を促す録音システムを提案する.
\end{jabstract}  % アブストラクト。要独自コマンド、include先参照のこと

\tableofcontents  % 目次
\listoffigures    % 図目次
% \listoftables    % 表目次

\pagenumbering{arabic}

\chapter{序論}
\label{chap:introduction}

本章では,本研究の背景と目的,および本論文の構成について述べる.

\newpage

\section{背景}

小型の録音装置は古くから販売されており,またスマートフォンでは様々な録音アプリケーションを利用できる.
これらは音楽に携わる人やインタビュー/会議を記録するといった用途に重宝されるが,
日常的にメモを記録するなどして活用している人は少ない.

\subsection{音声情報の有用性}

音声情報には,以下のような有用性が存在する.

\begin{enumerate}

\item 喋るだけで記録できる

2017年現在,ネット上での情報記録 / 情報交換はテキストベースのものがほとんどであるが,
録音ならPC / スマートフォンへのキー入力を行う必要がなく,記録したいことを喋るだけで良い.
日本の若者を中心に広く利用されているコミュニケーションサービスLINE\footnote{\textsf{https://line.me/ja/}}
ではボイスメッセージ機能が利用でき,その簡単さから一部のユーザーに積極的に活用されている.

\item 記録しながら同時に別のことを行える

長時間の録音でも,始めてしまえば記録に集中する必要がないので,同時に別のことを行える.
これはテキストや写真,動画による記録とは異なる点である.

\item 他のメディアとの相互補完性がある

また,写真や文章といった他のメディアとの組み合わせによって,
互いの内容理解を助ける相互補完性の存在が知られている\cite{Nakakura}.

\end{enumerate}

以上のように,音声情報は優れた情報記録 / 情報交換の手段となりうる可能性を秘めている.

\subsection{音声が活用されない理由}

音声の有用性が知られているにも関わらず多くの人に活用されないのは,便利に使える録音システムが存在しないためだと思われる.

既存の録音システムには,以下のような問題点が見受けられる.

\begin{enumerate}
\item 録音/再生の手順が煩雑

一般的に録音/停止/再生それぞれでボタン操作が必要であり,音声記録の単純さをインターフェースが阻害している.

\item 管理しづらい

録音時刻/再生時間といった基本的なメタデータや,自ら設定したタグ等をもとに音声を検索できるが,
音声が増えていくほど,内容理解を助ける仕組みとして十分とはいえない.

\item 音声を再利用しづらい

録音システムの中で音声を管理しなければならないことが多く,
外部アプリケーション / サービスでの音声活用の可能性を狭めている.

\end{enumerate}

音声を最大限活用するためには,これらの問題を解決する新しい録音システムが必要である.

\section{本研究の目的}

既存の録音システムに見られる不便を解消し,音声を有効活用できるようなシステムを開発することが本研究の目的である.

\section{本論文の構成}

第1章では,本研究における背景と問題意識,目的について述べた.
第2章では,第1章で述べた問題意識を踏まえ,新しい録音システムを提案する.
第3章ではシステムの実装に関して述べ,第4章では先行事例について述べる.
第5章ではシステムの考察を行い,第6章では本研究を総括する.  % 本文1
\chapter{システムの提案}
\label{chap:proposal}

本章では,音声利用についての背景を踏まえ、関連研究を概観し、新しい録音システム「Gyaon」を提案する。

\newpage

\section{関連研究およびアプリケーション}

\subsection{雰囲気の記録}

日常的な体験や雰囲気を記録するための録音デバイスについて研究が行われている\cite{Poupyrev}。
録音は何かを記録、説明したり、思い出すのに重要な手段だが、日常的に行っている人は少ないと指摘されている。
より気軽に録音できるよう日用品にセンサーを埋め込み、状況の変化に応じて録音/再生するデバイスが提案されている。
日用品に溶け込み、単純明快な録音/再生インターフェースが実現されているが、
本を閉じる/開くといった二値的な状況変化でしか操作できないため、
デバイス単体で複数の音声を管理することが困難になっている。

\subsection{長時間音声データの管理}

長時間にわたる音声データでは、聞き返したい箇所を効率良く見つけることが困難となる。
この課題に対して、音声データのブラウジングやタグ付けの手法について研究が行われている。

\paragraph*{ブラウジング}

人間生活に関わる情報を長期間にわたり記録するライフログ関連研究では、
そのログの一部として音声データがよく利用されている\cite{Bell}。
音声データの特徴量を計算し,閾値処理するインターフェースを提供することで再生区間を決定する手法\cite{Kawamura}や、
講義ノートといった手書きメモに対して音声データを関連付ける手法\cite{Stifelman}などが提案されている。

\paragraph*{タグ付け}

研究ノートの補助的な記録手段として音声データを利用するシステムが提案されている\cite{Kawanishi}。
事前に用意された研究ノートから形態素解析などを利用してキーワードを抽出し、音声データに付与する手法がとられている。

また、音声データの特定のタイミングにタグを付与できる録音システムが開発されている\cite{Fujisaka}。
学生がノートテイキングの補助として使うことを想定しており、重要な説明等を逃さないよう素早くタグ付けが行えるインターフェースが実装されている。

\subsection{音声ログの検索手法}

音声認識技術を活用したテキストによる音声検索手法が提案されている\cite{Vemuri}。
試作されたアプリケーションでは音声データに含まれる単語からキーワード検索が可能となっているほか、
音声認識の信頼度を文字色に反映させたり、ストップワードを見えにくくするなどして
ユーザが録音の要点を思い出しやすくなるよう配慮されている。

\subsection{写真との組み合わせ}
写真と音声を組み合わせによる体験記録手法が提案されている\cite{Nakakura}。
写真と手書きだけでは表現できない雰囲気を音声に記録し,
写真によって音声データに一覧性を持たせる仕組みである.
写真が音声の内容把握を助け,音声データの価値を高めることも確認されている。

また、同様の手法によって撮影された写真を、音声とともに閲覧できるWebサイトが公開されている\cite{Masui}。


\section{設計方針}

TBD

\section{利用例}

TBD

\chapter{実装}
\label{chap:implementation}

本章では,第2章で述べたシステムの設計を受け,Gyaonの実装について述べる.

\newpage

\section{システム構成}
Gyaonは,ユーザが実際に録音/再生するためのクライアントアプリケーションと,
アップロードされた音声を保存/管理するサーバから構成される.構成図を図\ref{system}に示す.

\begin{figure}[H]
\centering
\fbox{\includegraphics[width=15cm]{images/system.png}}
\caption{Gyaonシステムの構成図}
\label{system}
\end{figure}

\section{サーバ}
サーバはNode.js\footnote{\textsf{https://nodejs.org}}
とそのWebアプリケーションフレームワークであるExpress\footnote{\textsf{http://expressjs.com/ja/}}
によって実装されている.音声データの保存をAmazon S3
\footnote{\textsf{Amazon Web Servicesによって提供されるオンラインストレージサービス.https://aws.amazon.com/jp/s3/}},
DB管理をmLab\footnote{\textsf{MongoDBベースのクラウドDBサービス.https://mlab.com/}}にて行っている.

\subsection{DBスキーマ}

DBでは以下のようなスキーマを定義し,音声データを管理している.

\vspace{4mm}
\begin{lstlisting}
var soundSchema = mongoose.Schema({
    lastmodified: Date,     /* 録音時刻 */
    user: String,           /* ユーザID */
    name: String,           /* ファイル名 */
    key: String,            /* S3key */
    size: Number,           /* ファイルサイズ */
    time: Number,           /* 再生時間 */
    comment: String,        /* コメント */
    location_x: Number,     /* 位置情報 */
    location_y: Number
})
\end{lstlisting}

\section{PC版クライアント}
PC版クライアントはHTML/CSS/JavaScriptで実装されており,ブラウザ上のWebアプリケーションとして動作する.
%各機能の実装について述べる.
%特筆すべき実装
%詳細な実装

\subsection{WebAudioAPI}
%WebAudioAPIを解説する
録音/再生といった主要な音声機能は,
HTML5 audio\footnote{\textsf{https://www.w3.org/TR/html5/embedded-content-0.html\#the-audio-element}}
並びにWebAudioAPI\footnote{\textsf{https://www.w3.org/TR/webaudio/}}によって実装されている.

HTML5 audioはHTMLドキュメント内に音声を埋め込むことができ,以下のようなAudio要素を宣言することで利用できる.

\vspace{4mm}
\begin{lstlisting}
<audio src="/test/audio.mp3"></audio>
\end{lstlisting}

WebAudioAPIは高度な音声処理や音声合成を行えるJavaScriptAPIであり,
Audio要素を操作したり,エフェクトを加えることができる.

\subsection{音声リストの同期}
複数のユーザが同じユーザIDを使用していても,音声リストはリアルタイムに同期される.
これはNode.jsのWebSocketライブラリ「Socket.IO\footnote{\textsf{http://socket.io}}」によって実装されており,
サーバに音声がアップロードされると,音声リストの更新情報がブラウザに通知される.

%\subsection{プリレコーディング機能}
%shift/pushしている部分のコードを貼る

%録音ボタンを押す前からの音声を撮れる
%音声を入れてる配列の操作で
%〜は,以下のように実装されている
%プリレコーディング機能は,
%音声データを格納する配列へのshift/push
%配列を
%ことで実現されている.
%
%\vspace{4mm}
%\begin{lstlisting}
%preRecScriptProcessor.onaudioprocess = (event) => {
%    const channel = event.inputBuffer.getChannelData(0)
%    if(preRecAudioBufferArray.length * BUFFER_SIZE > SAMPLE_RATE * PREREC_SEC) {
%        preRecAudioBufferArray.shift()
%    }
%    preRecAudioBufferArray.push(new Float32Array(channel))
%}
%\end{lstlisting}

\subsection{Gyaonキー}
%Karabinerを使っている
%録音するシェルスクリプトを叩かせている
%スクリプト本体やkarabiner設定ファイルを貼る

Gyaonキーは,キーボードをカスタマイズするMacアプリケーション
「Karabiner\footnote{\textsf{https://pqrs.org/osx/karabiner/index.html.ja}}」
を利用して実装されている.
Karabinerを利用すると,キーボードが出力する文字列をカスタマイズしたり,
任意のキーでアプリを起動するよう指定できる.

筆者の環境では,command + fn キーで録音/アップロードを行うシェルスクリプトを実行するよう設定している.
以下にKarabiner設定ファイル(XML)の一部を示す.

\vspace{4mm}
\begin{lstlisting}
<vkopenurldef>
  <name>KeyCode::VK_OPEN_URL_SHELL_GYAON_START</name>
  <url type="shell">
    <![CDATA[    /bin/sh $HOME/.gyaon/start.sh    ]]>
  </url>
</vkopenurldef>
<item>
  <name>CMD_R + FN -> GYAON</name>
  <identifier>gyaon</identifier>
  <autogen>
    __SimultaneousKeyPresses__
    KeyCode::COMMAND_R, KeyCode::FN,
    KeyCode::VK_OPEN_URL_SHELL_GYAON_START,
    Option::NOREPEAT,
    Option::KEYTOKEY_AFTER_KEYUP,
    KeyCode::VK_OPEN_URL_SHELL_GYAON_STOP,
  </autogen>
</item>
\end{lstlisting}


\section{Android版クライアント}
Android版クライアントはJavaで実装されており,通常のAndroidアプリケーションとして動作する.
常駐型の録音ボタンは,バックグラウンド実行が可能なServiceとして実装されている.

%現状serviceを使っている
%常駐ボタンが邪魔だったりするので,他の方法を考えたい
%スマホを起動しなくてもいい方法
%    専用ボタンとか
%    ヘッドセットとか

\chapter{考察}
\label{chap:discussion}

本章では,「Gyaon」についての考察を行う。

\newpage

\section{自分で使ってみた}

TBD

\section{研究室で使ってみた}

TBD

\chapter{考察}
\label{chap:discussion}

本章では,Gyaonについての考察を行う。

\newpage

\section{単純な録音操作による音声利用の促進}
一般的な録音システムでは、録音開始/停止それぞれにボタン操作が必要であり、
外部で音声を利用する場合はPC等に音声ファイルをエクスポートしなければならない。
Gyaonでは録音ボタンを押している間録音し、停止後直ちに音声のアップロードが開始される手順となっており、単純な録音操作を実現している。

著者は開発時から継続的にGyaonを使用しており、単純な録音操作によって、録音行為に対する心理的障壁が低くなったと感じている。
このことから、以前まで手書きやキーボード入力していたメモなどを全て録音するようになり、音声を活用する場面が増えた。

\paragraph{研究会での活用}
著者の所属する増井研究会でも、Gyaonによって音声利用が促進された。

様々な素材の叩打音を分析する研究が行われており、そこではサンプリングのためにGyaonが利用された。
音声をすぐに確認でき、コメントを付けらる点が好評であった。

また大学が主催する研究発表イベントではリアルタイムな情報共有に利用された。
ブース展示がメインのイベントであったため、面白い展示を見つけた学生がGyaonで共有したり、
楽器のブースでは演奏を録音するなどした。
増井研究会のブースでは常にGyaonを起動していたので、有益な情報を見逃すことなく受け取ることができた。

%   共有しやすい[OK.icon]
%   	URLがあるので「これ聞け」がすぐできる
%
%   	長時間の音声を録るのはむずい
%   	おしっぱだから

%   PCもスマホも手元にない時はどうしようもない[NG.icon]
%   	録音機をばら撒くと解決するかも
%   今までのテキストメモに加えて音声メモも管理しないといけない[NG.icon]
%   	1か所に情報を溜めたいという欲求がある

\paragraph{雰囲気記録システムとしての活用}
スマートフォンでも簡単に録音できることから、その場の雰囲気を記録する用途に活用された。
鳥のさえずりや川のせせらぎといった環境音を記録して自宅で聞いたり、研究会のミーティングを録音し、
それを振り返りながらメンバーと懐かしむような場面もあった。

\paragraph{楽器練習支援システムとしての活用}
Gyaonペダルとプレビュー再生機能の組み合わせによって、楽器練習支援システムとして便利に活用できた。
楽器演奏において自分の演奏を客観的に評価することは難しいが、録音を行うことによって解決される。
Gyaonペダルを利用すれば演奏中でも録音でき、録音後のプレビュー再生で演奏をすぐに確認することができる。
演奏/録音/確認を素早く行えることから、従来の録音機では実現できなかった新しい練習形態を生み出したといえる。

%   さっき撮った音をもう一度聞きたくなった時は大変[NG.icon]
%   	楽器を持っていて手が空いてない前提
%    画面がなくても操作できれば良い?


\section{外部システムでの活用}
%エクスポート
%URL発行
%URL取得
%URLによるアクセス
%の重要性
%
%Webに貼れることは重要
%音声のURLがあることは重要
%
%    研究室で使ってもらった
%    Scrapbox
%    オーディオ記法というものが使える
%   	文章に音声を埋め込める
%    	GyaonのURLを貼る
%    GyaonWebと同様マウスオーバーで簡単に再生可能
%    URLだけで利用できるので親和性高し
%    活用例
%     [何か書くと早川が録音して返す]
%     	ゼミの時間におもしろ音声を流して盛り上がった
%      音声素材置場として活用できた
%      合成音声として使うというアイデアが生まれた
%     [Gyaon/ORF2016音声]
%     	ポスター発表中にもらった意見を忘れず記録できた
%      Wikiに貼ることで研究室のメンバーに共有できる
%      いつでも見返せる
%     [図形に関する英単語]
%     	発音をすぐ再生できる
%      単語に関する画像なども埋め込んでおけば強く印象に残るハズ
%
% 〜という機能を付けたら〜という使い方を発見して幸せになった!

\section{プライバシ問題}
% もともといろいろ指摘されてた\cite{}

\begin{acknowledgment}

本論文の執筆において,担当教員である増井俊之教授に多大なるご指導と貢献をしていただきました.
また,本システムの実装において,研究会OBの橋本 翔氏,桜井雄介氏に多くの貢献と助言を頂きました.
環境情報学部の早川 匠氏には有用な音声素材を提供していただきました.
この場を借りて感謝の意を表します.

\end{acknowledgment}
  % 謝辞。要独自コマンド、include先参照のこと
\include{91_bibliography}  % 参考文献。要独自コマンド、include先参照のこと
\appendix
%\include{92_appendix}    % 付録

\end{document}
